\graphicspath{{chapters/02/images}}
\chapter{Smith Watermann}

\section{Introduction}
The Smith Watermann algorithm extends the one of Needleman and Wunsch to find a pair of segment, one from each of two long sequences, such that there is no other pair of segments with greater similarity.
This similarity measure allows for deletion and insertion of arbitrary length.

\section{Algorithm}
Consider two molecular sequences $A = a_1a_2\dots a_n$ and $B = b_1b_2\dots b_m$.
Given a similarity $s(a,b)$ of elements of the sequence and $W_k$ the weight of deletions of length $k$, to find pairs of segments with high degrees of similarity, a matrix $H$ is set up such that:

$$H_{k0} = H_{0l} = 0\qquad \forall 0\le k\le n\land 0\le l\le m$$

$H_{ij}$ is the maximum similarity of two segments ending in $a_i$ and $b_j$.
$H_{ij}$ is computed such that:

$$H_{ij} = \max(H_{i-1, j-1} + s(a_i, b_j), \max\limits_{k\ge 1}(H_{i-k, j}-W_k), \max\limits_{l\ge 1}(H_{i, j-l}-W_k), 0)$$

With $1\le i\le n$ and $1\le j\le m$.
So $H_{ij}$ is:

\begin{multicols}{2}
	\begin{itemize}
		\item $H_{i-1, j-1} + s(a_i, b_j)$ If $a_i$ and $b_j$ are associated.
		\item $H_{i-k, j}-w_k$ if $a_i$ is at the end of a deletion of length $k$.
		\item $H_{i-k, j}-W_l$ if $b_j$ is at the end of a deletion of length $l$.
		\item $0$ is used to prevent calculated negative similarity, indicating no similarity up to $a_i$ and $b_j$.
	\end{itemize}
\end{multicols}

The pair of segments with maximum similarity is found first by locating the maximum element of $H$.
The other elements are determined sequentially with a traceback procedure ending with an element of $H$ equal to $0$.
This procedure other than identifying the elements produces their alignment.
The parameters  where:

$$s(a_i, b_j) = \begin{cases} 1 & a_i = b_j \\ 0 & a_i\neq b_j\end{cases}$$

And

$$W_k = \frac{1}{3}k$$

This algorithm in particular allows for the alignment of sequences that contained both mismatches and internal deletions.
