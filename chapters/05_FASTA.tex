\chapter{FASTA}

\section{Introduction}

	\subsection{Abstract}
	FASTA is a derivative of FASTP and can be used to search protein or DNA sequence data bases and compare a protein sequence to a DNA one by translating the DNA one as it is searched.
	FASTA includes an additional step in the calculation of the initial pairwise similarity score that allows multiple regions of similarity to be joined to increase the score of related sequences.
	RDF2 can be used to evaluate the significance of similarity score with a shuffling method that preserves local sequence composition.
	LFASTA can display all the regions of local similarity with scores greater than a threshold that can be displayed a graphic matrix plot.

	\subsection{Introduction}
	There is a trade-off between sensitivity and selectivity in biological sequence comparison: methods that can detect more distantly related sequences (increased sensitivity) frequently increase the similarity scores of unrelated sequences (decreased selectivity).
	FASTA use an improved algorithm that increases sensitivity with a small loss of selectivity and a negligible decrease in speed.
	LFASTA compute local similarity analyses of DNA or amino acid sequences.

\section{Methods}

	\subsection{The search algorithm}
	The search algorithm has four steps in determining a score for pair-wise similarity.

		\subsubsection{First step - lookup table}
		FASTA achieve its speed and selectivity in this step using a lookup table to locate all identities or groups of identities between two sequences.
		The $ktup$ parameter determines how many consecutive identities are required in a match.
		The $10$ best diagonal regions are found using a simple formula based on the number of $ktup$ matches and the distance between the matches without considering shorter runs of identities, conservative replacements, insertions or deletions.

		\subsubsection{Second step - rescoring}
		In the second step the $10$ regions are rescored using a scoring matrix that allows conservative replacements and runs of identities shorter than $ktup$ to contribute to the similarity score.
		For each of these best diagonal regions, a subregion with minimal score is identified.
		This is the initial region.

		\subsubsection{Third step - ranking}
		The best scoring initial region is used to characterized pair-wise similarity and the initial scores are used to rank the library sequences.
		FASTA checks if several initial regions may be joined together.
		Given the location, their scores and a joining penalty, FASTA calculates an optimal alignment of initial regions as a combination of compatible regions with maximal score.
		The resulting score is used to rank the library sequences.
		Only for regions whose scores are above a threshold are used in this process.

		\subsubsection{Fourth step - alignment}
		The highest scoring library sequences are aligned by a modification of the Needleman-Wunsch and Smith-Waterman.
		The final comparison considers all possible alignments of the query and library that fall within a band centered around the highest scoring initial region.

	\subsection{Local similarity analyses}
	The program for detecting local similarities, LFASTA uses the same two steps for finding initial regions.
	Instead of saving $10$, all diagonal regions with scores greater than a threshold are saved.
	Instead of focusing on a single region, LFASTA computes a local alignment for each initial region.
	It considers all initial regions and potential sequence alignments for some distance before and after the initial region.
	Starting at the end of the initial region, an optimization proceeds in the reverse direction until all possible alignment scores have gone to zero.
	The maximal local similarity score in the reverse direction is used to start a second optimization that proceeds in the forward direction.
	An optimal path starting from the forward maximum is displayed.
	The local homologies can be displayed as sequence alignments or on a two-dimensional graphic matrix sty.
	The maximal local similarity score in the reverse direction is used to start a second optimization that proceeds in the forward direction.
	An optimal path starting from the forward maximum is displayed.
	The local homologies can be displayed as sequence alignments or on a two-dimensional graphic matrix style plot

	\subsection{Statistical significance}
	To evaluate the statistical significance of an alignment RDF2 is used.
	RDF2 calculates three scores for each shuffled sequence: one from the best single initial region, a second from the joined initial regions and a third from the optimized diagonal.
	RDF2 can be used to evaluate amino acid or DNA sequences and allows to specify the scoring matrix.
	Moreover a global or local shuffle routine can be specified.
	Locally biased composition is the most common reason for high similarity scores of dubious biological significance.
	A Monte Carlo shuffle analysis that constructs random sequences by taking each residue in one sequence and placing it randomly along the length of the new sequence will break up these patches of biased composition.
	The scores may be lower and the sequences will appear to be related.
