\graphicspath{{chapters/04/images}}
\chapter{BLOSUM}

\section{Introduction}

	\subsection{Abstract}
	The most used substitution matrix with scores for all possible exchanges of one amino acid is based on the Dayhoff model of evolutionary rates.
	This work proposes a different approach from blocks of aligned sequence segments, leading to improvement in alignments and in searches.

	\subsection{Introduction}
	Sequence alignment of proteins provide important insights into gene and protein function.
	There are different types of alignments:

	\begin{multicols}{2}
		\begin{itemize}
			\item Global alignments of pairs related by common ancestry.
			\item Multiple alignments of members of protein families.
			\item Alignments made during data base searches to detect homology.
		\end{itemize}
	\end{multicols}

	In each case a scoring scheme for estimating similarity is used.
	The mutation data matrices of Dayhoff are considered the default in alignment and searching programs.
	However the most common task is the detection of much more distant relationships, which are only inferred from substitution rates in the Dayhoff model.

\section{Methods}

	\subsection{Deriving a frequency table from a data base blocks}
	Local alignments can be represented as ungapped blocks with each row a different protein segment and each column an aligned residue position.
	Protomat can be used for obtaining a set of blocks given a group of related proteins.
	Considering a single block representing a conserved region of a protein family, for a new member a set of score for matches and mismatches that best favours a correct alignment with each of the other segments in the block.
	For each column of the block, the number of matches and mismatches of each type between the new sequence and every other are counted.
	This is repeated for all columns of all blocks and the summed results are stored in a table.
	The new sequence is then added to the group.
	For another new sequence the procedure is repeated.
	Doing so the table in the end will consist of counts of all possible amino acid pairs in a column.
	Counts of all possible pairs in each column of each block in the data base are summed.
	If a block has a width $w$ amino acids and a depth of $s$ sequences, contributes $ws\frac{(s-1)}{2}$ amino acids pairs to the count.
	This results in a frequency table listing the number of times each of the different amino acid pairs occurs among the blocks.
	This table is used to compute the odds ratio matrix between the observed frequencies and the expected one.

	\subsection{Computing a logarithm of odds matrix}
	Let the total number of amino acid pairs $i,j$ for each entry of the frequency table be $f_{ij}$.
	Then the observed probability of occurrence for each $i,j$ pair is:

	$$q_{ij} = \frac{f_{ij}}{\sum\limits_{i=1}^{20}\sum\limits_{j=1}^if_{ij}}$$

	The expected probability of occurrence for each pair is computed following the occurrence of the $ith$ amino acid in a $i,j$ pair:

	$$p_i = q_{ii}+\sum\limits_{j\neq i}\frac{q_{ij}}{2}$$

	The expected probability of occurrence $e_{ij}$ for each $i,j$ pair is then $p_ip_j$ for $i=j$ and $p_ip_j + p_jp_i = 2p_ip_j$ for $i\neq j$.
	An odds ratio matrix is computed such that each entry is $\frac{q_{ij}}{e_{ij}}$.
	A lod (logarithm of odds) is then calculated in bit units as $s = \log_2\big(\frac{q_{ij}}{e_{ij}}\big)$.
	If the observed frequencies are as expected $s_{ij} = 0$, if less $s_{ij} < 0$, if more $s_{ij} > 0$.
	Lod ratios are multiplied by a $2$ scaling factor and rounded to the nearest integer value to produce the block substitution matrix BLOSUM.
	The relative entropy or the average mutual information per amino acid pair is computed:

	$$ H = \sum\limits_{i=1}^{20}\sum\limits_{j=1}^i q_{ij}\times s_{ij}$$

	And the expected score in bit units:

	$$E = \sum\limits_{i=1}^{20}\sum\limits_{j=1}^i p_i\times p_j\times s_{ij}$$

	\subsection{Clustering segments within blocks}
	To reduce multiple contributions to amino acid pair frequencies from the most closely related members of a family, sequences are clustered within blocks and each cluster is weighted as a single sequence in counting pairs.
	A clustering percentage in which sequence segments identical for at least that value are clustered is used.
	The contribution of closely related segments to the frequency table is reduced.
	Varying the clustering percentage leads to a family of matrices.

	\subsection{Constructing blocks data bases}
	Protomat was used to build the block data base from $504$ non redundant groups of proteins.
	Protomat uses an amino acid substitution matrix at two phases.
	The motif program uses a substitution matrix when individual sequences are aligned against sequence segments containing a candidate motif.
	The Motomat program uses a substitution matrix when a block is extended to either side of the motif region.
	A unitary substitution matrix was used, next blosum was applied to the blocks and the resulting matrix was used to construct a second database.
	Then blosum was applied to the second data base and the resulting matrix was used to construct version of blocks data base.
	The blosum program was applied to the final data base using a series of clustering percentages to obtain a family of lod substitution matrices.
	Similar matrices were obtained using PAM.

	\subsection{Alignments and homology searches}
	Global multiple alignments were done using Multalin and to provide a positive matrix each entry was increased by $8$.
	Pearson's RDF2 program was used to evaluate local pairwise alignments.
	Homology searches were done using Blastp, Fasta and Ssearch.
	The Swiss-Prot data bank was searched.
	The first of the longest and most distance sequences in the group was used as a searching query, inferring distance from Protomat results.
	The results of each search were analysed by considering the sequences used by Protomat to construct blocks for the protein group as the true positive sequences.
	The number of misses is the nubmer of true positive sequences not reported for blastp.
	For fasta and ssearch the empirical evaluation criteria of Pearson was used: the number of misses is the number of true positive scores which ranked below the $99.5$ percentile of the true negative scores.

\section{Results}

	\subsection{Comparison to Dayhoff matrices}
	The blosum series based on percent clustering can be compared to the Dayhoff matrices using a measure of average information per residue pair in bit units or relative entropy, which is $0$ when the target distribution of pair frequencies is the same as the expected one and increases as they become more different.
	In the Dayhoff matrices relative entropy decreased when increasing PAM, while in blosum it increased linearly when increasing clustering percentage.
	Matrices with comparable relative entropy have similar expected scores.

	\subsection{Performance in multiple alignment of known structures}
	To test sequence alignment accuracy the results obtained to alignments seen in three dimensional structures was used.
	A simultaneous multiple alignment program MSA was used as a standard.
	Multalin, a hierarchical multiple alignment program performed worse using PAM matrices, while using blosum better.
	Therefore blosum matrices produced accurate global alignments of these sequences.

	\subsection{Performance in searching for homology in sequence data banks}
	The number of misses when searching was averaged in order to assess the overall searching performance of different matrices using blast, fasta and smith-waterman.
	Blosum matrices performed better than the best PAM matrix.
	BLOSUM improved detection of members of this family.
	The test was repeated with similar result of PAM for other protein families.
